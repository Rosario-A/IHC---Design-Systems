\documentclass[11pt]{beamer}
\usepackage{listings} % Include the listings-package
\usepackage[T1]{fontenc}
\usepackage[utf8]{inputenc}
\usepackage[english]{babel}
\usepackage{amsmath}
\usepackage{amssymb, amsfonts, latexsym, cancel}
\usepackage{float}
\usepackage{graphicx}
\usepackage{epstopdf}
\usepackage{subfigure}
\usepackage{hyperref}
\usepackage{blindtext}
\usepackage{booktabs} % Allows the use of \toprule, 
\usepackage{filecontents}
\usepackage{courier} %% Sets font for listing as Courier.
\usepackage{listings}
\usepackage{ragged2e}

\usepackage{media9}

\usepackage{animate}

%\usepackage{listings, xcolor}
\lstset{
tabsize = 2, %% set tab space width
showstringspaces = false, %% prevent space marking in strings, string is defined as the text that is generally printed directly to the console
numbers = left, %% display line numbers on the left
commentstyle = \color{green}, %% set comment color
keywordstyle = \color{blue}, %% set keyword color
stringstyle = \color{red}, %% set string color
rulecolor = \color{black}, %% set frame color to avoid being affected by text color
basicstyle = \small \ttfamily , %% set listing font and size
breaklines = true, %% enable line breaking
numberstyle = \tiny,
}
\usepackage{caption}
\DeclareCaptionFont{white}{\color{white}}
\DeclareCaptionFormat{listing}{\colorbox{gray}{\parbox{\textwidth}{#1#2#3}}}
\captionsetup[lstlisting]{format=listing,labelfont=white,textfont=white}
\definecolor{urlColor}{rgb}{0.06, 0.3, 0.57}
\definecolor{linkColor}{rgb}{0.57, 0.0, 0.04}
\definecolor{fileColor}{rgb}{0.0, 0.26, 0.26}
\hypersetup{
    colorlinks=true,
    linkcolor=linkColor,
    filecolor=fileColor,      
    urlcolor=urlColor,
}
\urlstyle{same}
\setbeamercovered{transparent}

\usetheme{CambridgeUS}


\title[]{\bf\Huge Taller: Design Systems}
\subtitle{Interacción Humano Computador}

\author[rescobedoq]
{
    INTEGRANTES:
	\newline Gabriel Callo Condori 
	\newline José Rodríguez Mercado
	\newline José Miguel Vera Mamani
	\newline Rosario Lohana Alegre Linares
}
\institute[UNSA]
{
\inst{1}% 
System Engineering School\\
System Engineering and Informatic Department\\
Production and Services Faculty\\
San Agustin National University of Arequipa
}

\date[2020-10-06]{\scriptsize{2020-10-06}}
\logo{\includegraphics[width=3.0cm]{logo_unsa.jpg}}
%\titlegraphic{\includegraphics[width=4.5cm]{logo_unsa.jpg}}

\begin{document}

\begin{frame}
\titlepage
\end{frame}

\begin{frame}
\frametitle{Contenido}
\tableofcontents
\end{frame}

\section{Definición}
\begin{frame}
\frametitle{Definición}
\begin{itemize}
\item Existen varias definiciones como por ejemplo: uxpin.com, medium.muz.li, Freecodcamp.org, css-tricks.com.\newline
\item De todas ellas se puede inferir la siguiente definición: \newline{\it "Es un ecosistema \textbf{dinámico, escalable y ordenado} de elementos gráficos, código y documentación, para la construcción de cualquier producto.\\Donde interactúan, aprenden y se toman decisiones para \textbf{todas las partes involucradas} en la construcción del producto."}
\end{itemize}
\end{frame}

\section{Elementos Básicos}
\begin{frame}
\frametitle{Elementos Básicos}
\begin{itemize}
\item Principios de Diseño
\item Tono de Voz
\item Colores y Tipografía
\item Componentes de Diseño
\end{itemize}
\justifying \vspace{5mm} Hay muchos otros, pero dependen de los pensamientos de la empresa.
\begin{figure}
  \centering
  \includegraphics[width=0.4\textwidth]{sd_elements.png} 
\end{figure}
\end{frame}

\section{Principios de Diseño}
\begin{frame}
\frametitle{Principios de Diseño}
\justifying \vspace{5mm} Son \textbf{paradigmas} que dirigen las \textbf{decisiones del diseño}.\newline\addlinespace
El concepto de Principios de Diseño no se refiere a los \textbf{fundamentos universales del diseño} sino a como el equipo de desarrollo \textbf{afronta} las decisiones del diseño (Es como establecer \textbf{reglas} del actuar de la empresa de manera gráfica).
\begin{figure}
  \centering
  \includegraphics[width=0.4\textwidth]{sd_pd.jpg} 
\end{figure}
\end{frame}

\begin{frame}

\newline \large {\bf Medium}
\newline Herramienta con muchos artículos de programación. \newline Principios de diseño:\addlinespace
\begin{minipage}[c]{0.55\textwidth} 
\begin{itemize}
    \item {\bf Dirección} sobre opciones\newline (dirección clara)
    \item {\bf Apropiado} sobre consistencia\newline (facilidad)
    \item {\bf En evolución} más que finalizado\newline (dinamismo)
\end{itemize}
\end{minipage} 
\begin{minipage}[c]{0.40\textwidth} 
\includegraphics[width=4.5cm, height=3.5cm]{medium.jpeg} 
\end{minipage}
\end{frame}

\begin{frame}
\begin{minipage}[c]{0.35\textwidth} 
\includegraphics[width=3cm]{apple.png} 
\end{minipage}
\begin{minipage}[c]{0.60\textwidth} 
{\Large \bf Apple}\newline Principios de diseño:\addlinespace
\begin{itemize}
    \item {\bf Integridad Estética:} Una visualización completa.
    \item {\bf Consistencia:} Por el alcance de Apple.
    \item {\bf Manipulación Directa}
    \item {\bf Feedback} (Retroalimentación)
    \item {\bf Metáforas}
    \item {\bf Control Visual}
\end{itemize}
\end{minipage} 
\end{frame}

\begin{frame}
\newline {\Large \bf Asana}\newline Principios de diseño:\addlinespace
\begin{minipage}[c]{0.55\textwidth} 
\begin{itemize}
    \item Permite enfocarse en el \textbf{trabajo} y no en la \textbf{interfaz}.
    \item Incrementa la confianza a través de la \textbf{claridad}.
    \item Promueve la \textbf{dinámica interpersonal} y la \textbf{productividad}.
\end{itemize}
\end{minipage}
\begin{minipage}[c]{0.30\textwidth} 
\includegraphics[width=3.5cm, height=2.5cm]{asana.png} 
\end{minipage}
\end{frame}

\section{Tono de Voz}
\begin{frame}
\frametitle{Tono de voz}
\justifying \vspace{5mm} Es la \textbf{forma} en que la marca o producto se \textbf{comunica} con sus usuarios.\newline
\begin{figure}
  \centering
  \includegraphics[width=0.5\textwidth]{tono.jpg} 
\end{figure}
\end{frame}

\begin{frame}
\begin{minipage}[c]{0.55\textwidth} 
\begin{itemize}
    \item We are plainspoken (hablamos \textbf{claro}).
    \item We are genuine (No se guían de \textbf{patrones establecidos} en lo posible).
    \item We are translators (No usar \textbf{lenguaje técnico}).
    \item Our Humor is dry (Usar sentido del \textbf{humor} en el momento apropiado).
\end{itemize}
\end{minipage}
\begin{minipage}[c]{0.30\textwidth} 
\includegraphics[width=5cm]{mail_chimp.png} 
\end{minipage}
\end{frame}


\begin{frame}
\includemedia[width=12cm, height=7cm,activate=pageopen,
passcontext,
transparent,
addresource=acciones.mp4,
flashvars={source=acciones.mp4}
]{\includegraphics[width=12cm, height=7cm]{acciones.jpg}}{VPlayer.swf}
\end{frame}

\section{Colores y Tipografías}
\begin{frame}
\frametitle{Colores y Tipografías}
Características Básicas
\begin{itemize}
\item COLORES
\end{itemize}

\begin{minipage}[c]{0.6\textwidth} 
\includegraphics[width=7cm, height=3.5cm]{gov_uk.jpg} 
\end{minipage}
\begin{minipage}[c]{0.3\textwidth} 
\includegraphics[width=4.7cm, height=3.5cm]{material_design.JPG} 
\end{minipage} 
\end{frame}

\begin{frame}
\frametitle{Paletas de Accesibilidad}
\begin{minipage}[c]{0.5\textwidth} 
\includegraphics[width=6cm, height=6cm]{shopify.JPG} 
\end{minipage}
\begin{minipage}[c]{0.4\textwidth} 
\includegraphics[width=6cm, height=5.7cm]{IBM.JPG} 
\end{minipage} 
\end{frame}

\section{Componentes de Diseño}
\begin{frame}
\begin{itemize}
\item Colección de Elementos
\item Múltiples Veces
\end{itemize}
\frametitle{Componentes de Diseño}
\begin{minipage}[c]{0.45\textwidth} 
\includegraphics[width=5.5cm, height=2.5cm]{adobe.JPG} 
\newline \newline \newline \newline \newline \newline \newline \newline
\end{minipage}
\begin{minipage}[c]{0.3\textwidth} 
\includegraphics[width=6.5cm, height=5cm]{audi.JPG} 
\newline \newline \newline
\end{minipage} 
\end{frame}

\section{Guía de Diseño}
\begin{frame}
\begin{itemize}
\item Importante Documentar
\end{itemize}
\frametitle{Guía de Diseño}
\includegraphics[width=12cm, height=4.8cm]{guia_de_ilustración.JPG} 
\newline \newline \newline 
\end{frame}

\section{Guías de Onboarding, Mobile Onboarding y Animaciones}
\begin{frame}
\frametitle{Guías de Onboarding, Mobile Onboarding y Animaciones}
\includemedia[width=12cm, height=7.5cm,activate=pageopen,
passcontext,
transparent,
addresource=carga.mp4,
flashvars={source=carga.mp4}
]{\includegraphics[width=12cm, height=7.5cm]{carga.JPG}}{VPlayer.swf}
\newline
\end{frame}

\section{Micro Interacciones}
\begin{frame}
\frametitle{Micro Interacciones}
\includemedia[width=12cm, height=7.5cm,activate=pageopen,
passcontext,
transparent,
addresource=interacciones.mp4,
flashvars={source=interacciones.mp4}
]{\includegraphics[width=12cm, height=7.5cm]{intera.jpg}}{VPlayer.swf}
\newline
\end{frame}

\section{Estilos de Diseño}
\begin{frame}
\frametitle{Estilos de Diseño}
\begin{minipage}[c]{7cm} 
\includegraphics[width=7cm, height=6cm]{estilo.JPG} 
\end{minipage}
\begin{minipage}[c]{4cm} 
Realism Sketmorphic
\begin{itemize}
\item Sombras
\item Texturas
\item Detalles
\end{itemize}
Flat Design
\begin{itemize}
\item Claridad
\item Sencillez
\item Diseño Responsive (adaptabilidad)
\end{itemize}
\end{minipage} 
\end{frame}

\section{Diseño Atómico}
\begin{frame}
\frametitle{Diseño Atómico}
\begin{minipage}[c]{4cm} 
Es un sistema de trabajo que se basa en la creación de elementos modulares sencillos para crear estructuras de información mucho mas complejas.
\begin{figure}[posición]
  \centering
  \includegraphics[width=3cm, height=2cm]{biblioteca.JPG} 
\end{figure}
\end{minipage}
\begin{minipage}[c]{7cm} 
\includegraphics[width=8cm, height=5cm]{emerge.JPG}
\end{minipage} 
\end{frame}

\section{Bibliotecas de Componentes}
\begin{frame}
\frametitle{Bibliotecas de Componentes}
\includegraphics[width=12cm, height=6cm]{componentes.JPG} 
\newline \newline
\end{frame}

\section{Reglas de Diseño}
\begin{frame}
\frametitle{Reglas de Diseño}
\includegraphics[width=12cm, height=5cm]{reglas.JPG} 
\newline \newline
\end{frame}

\section{Reglas de Comunicación Visual}
\begin{frame}
\frametitle{Reglas de Comunicación Visual}
\begin{minipage}[c]{6cm} 
\begin{itemize}
\item Armonía
\end{itemize}
\begin{figure}[posición]
  \centering
  \includegraphics[width=5cm, height=4cm]{armonia.JPG} 
\end{figure}
\end{minipage}
\begin{minipage}[c]{5.5cm} 
\begin{itemize}
\item Ritmo
\end{itemize}
\begin{figure}[posición]
  \centering
  \includegraphics[width=5cm, height=4cm]{ritmo.JPG} 
\end{figure}
\end{minipage} 
\end{frame}


\begin{frame}
\includegraphics[width=12cm, height=6cm]{comunicacion.JPG} 
\end{frame}

\section{Ejemplos en Producción}
\begin{frame}
\frametitle{Ejemplos en Producción}
\begin{minipage}[c]{12cm} 
    \begin{minipage}[c]{6cm} 
    \begin{itemize}
    \end{itemize}
    \begin{figure}[posición]
      \centering
      \includegraphics[width=5cm, height=3cm]{material.JPG} 
    \end{figure}
    \end{minipage}
    \begin{minipage}[c]{5cm} 
    {\bf Material Design:} Es un lenguaje visual que sintetiza los principios clásicos del buen diseño con la innovación de la tecnología y la ciencia
    \end{minipage} 
\end{minipage}

\begin{minipage}[c]{12cm} 
    \begin{minipage}[c]{6cm} 
    \begin{itemize}
    \end{itemize}
    \begin{figure}[posición]
      \centering
      \includegraphics[width=5cm, height=3cm]{repo.JPG} 
    \end{figure}
    \end{minipage}
    \begin{minipage}[c]{5cm} 
    {\bf Design Systems Repo:} Una lista completa y seleccionada de sistemas de diseño, guías de estilo y bibliotecas de patrones que puede ser utilizada como inspiración.
    \end{minipage} 
\end{minipage}
\end{frame}

\section{Referencias}
\begin{frame}
\frametitle{Referencias}
\begin{itemize}
\item Taller: Design Systems, Carlos Arbieto (2020). www.youtube.com/watch?v=meO8gkOK1pc&feature=youtu.be
\item Design process simplified - https://www.uxpin.com/
\item Muzli Magazine - https://medium.muz.li/?gi=95d2e01562dd
\item Free Code Camp - https://www.freecodecamp.org/
\item CSS Tricks - https://css-tricks.com/
\item Figma - https://www.figma.com/
\end{itemize}
\end{frame}

\end{document}
